
\chapter{Оценка эффективности алгоритма} \label{chapt4}


  Существует проблема в виде отсутствия общепринятого корпуса математических выражений, разбитых по определенным классам схожести. Это вызывает трудности для проверки эффективности разработанного алгоритма на основе определенных в \ref{chapt1_simPatterns} паттернов схожести математических выражений. Ниже, в подразделе \ref{chapt4__corpus} будут приведены свои корпуса, а в подразделах \ref{chapt4__efficiency__struct} и \ref{chapt4__efficiency__subexpr} будет рассмотрена эффективность разработанного алгоритма на их основе.

  % ПРО ОБЕЗЛИЧИВАНИЕ ПЕРЕМЕННЫХ РАССКАЗАТЬ!


	\section{Составление корпусов для оценки эффективности алгоритма} \label{chapt4__corpus}


    За источник математических выражений, взятых в основу для составления корпусов, был взят образец 13-го задания ЕГЭ профильного уровня по математике, содержащий тригонометрические уравнения. Набор из 30 заданий из этого образца продемонтрирован в табл.~\ref{tbl:baseEGE}. 
    
      \begingroup
      \footnotesize
      \renewcommand\arraystretch{1.4}
      \begin{center}
      \begin{longtable}{|c|c||c|c|}
      \caption{База уравнений}\label{tbl:baseEGE}\\
      \hline
      № & Выражение & № & Выражение\\
      \hline
      \endfirsthead
      \multicolumn{4}{l}%
      {Продолжение табл. \thetable\ } \\
      \hline
      № & Выражение & № & Выражение\\
      \hline
      \endhead
      \hline
      \endfoot
      \hline
      \endlastfoot
            1   & $\sqrt{2}\sin (\frac{3\pi}{2} - x)\sin x = \cos x$      & 
            16 & $2\cos (\frac{\pi}{2} + x) = \sqrt{3}\tan x$             \\ 

            2 & $\cos (\frac{\pi}{2} + 2x) = \sqrt{2}\sin x$              &
            17 & $\sin 2x + 2\sin^2 x = 0$                                \\

            3 & $2\cos (x - \frac{11\pi}{2})\cos x = \sin x$              &
            18 & $2\sin (\frac{7\pi}{2} - x)\sin x = \cos x$              \\

            4 & $2\sin^ 4x + 3\cos 2x + 1 = 0$                            &
            19 & $2\sin^2 x - \sqrt{3}\sin 2x = 0$                        \\

            5 & $(2\cos x + 1)(\sqrt{-\sin x} - 1) = 0$                   &
            20 & $\cos 2x - 3\cos x + 2 = 0$                              \\

            6 & $(2\sin x - 1)(\sqrt{-\cos x} + 1) = 0$                   &
            21 & $2\cos^3 x - \cos^2 x + 2\cos x - 1 = 0$                 \\

            7 & $4 \sin^4 2x + 3\cos 4x - 1 = 0$                          &
            22 & $\cos 2x + 3\sin x - 2 = 0$                              \\

            8 & $\cos 2x = \sin (x + \frac{\pi}{2})$                      &
            23 & $\sin 2x + \sqrt{2}\sin x = 2\cos x + \sqrt{2}$          \\

            9 & $ 2\sqrt{3}\cos^2 (\frac{3\pi}{2}+x) - \sin 2x = 0$       &
            24 & $3\cos 2x - 5\sin x + 1 = 0$                             \\

            10 & $\cos^2 x - \frac{1}{2}\sin 2x + \cos x = \sin x$        &
            25 & $\cos 2x - 5\sqrt{2}\cos x - 5 = 0$                      \\

            11 & $\cos 2x = 1 - \cos(\frac{\pi}{2} - x)$                  &
            26 & $-\sqrt{2}\sin (-\frac{5\pi}{2} + x)\sin x = \cos x$     \\

            12 & $\sqrt{\cos^2x - \sin^2 x}(\tan 2x - 1) = 0$             &
            27 & $\frac{2\sin^2 x - \sin x}{2\cos x - \sqrt{3}} = 0$      \\

            13 & $\tan x + \cos (\frac{3\pi}{2} - 2x) = 0$                &
            28 & $\frac{2\sin^2 x - \sin x}{2\cos x + \sqrt{3}} = 0$      \\

            14 & $\cot x + \cos (\frac{\pi}{2} + 2x) = 0$                 &
            29 &  $4\cos^4 x - 4\cos^2 x + 1 = 0$                         \\

            15 & $\frac{1}{2} \sin 2x + sin^2x - \sin x = \cos x$         &
            30 & $4\sin^2 x + 8\sin(\frac{3\pi}{2} + x) + 1 = 0$          \\
      \end{longtable}
      \end{center}
      \normalsize
      \endgroup



    Задания были заимствованы с общеобразовательного портала для подготовки абитуриентов \cite{reshuege}. На основе этих уравнений в составе двух экспертов в области подготовки к единому государственному экзамену --- автором данной магистерской работы и одним из преподавателей в университете --- были составлены:

      \begin{itemize}
        \item корпус, разбивающий рассматриваемые уравнения по классам структурной схожести (табл. \ref{tbl:Corpus_structSim});
        \item корпус, разбивающий уравнения по классам схожести подвыражений (табл. \ref{tbl:Corpus_subexprSim}). 
      \end{itemize}


    В ходе составления корпусов, разногласий по отнесению уравнений к тому или иному виду схожести не было.
    Ниже приведены составленные корпуса.

      \begingroup
      \footnotesize
      \renewcommand\arraystretch{1.4}
      \begin{center}
      \begin{longtable}{|c|c|c|}
      \caption{Корпус, разбивающий уравнения по классам структурной схожести}\label{tbl:Corpus_structSim}\\
      \hline
      № & Выражение & Класс структурн. схож.\\
      \hline
      \endfirsthead
      \multicolumn{3}{l}%
      {Продолжение табл. \thetable\ } \\
      \hline
      № & Выражение & Класс структурн. схож.\\
      \hline
      \endhead
      \hline
      \endfoot
      \hline
      \endlastfoot
            1   & $\sqrt{2}\sin (\frac{3\pi}{2} - x)\sin x = \cos x$         & \circled{1}\\ 
            
            2   & $2\cos (x - \frac{11\pi}{2})\cos x = \sin x$               &\\ 
            
            3  & $2\sin (\frac{7\pi}{2} - x)\sin x = \cos x$                 &\\ 
            
            4  & $-\sqrt{2}\sin (-\frac{5\pi}{2} + x)\sin x = \cos x$        &\\ 
            \hline
            


            5  & $\cos 2x - 3\cos x + 2 = 0$                                 & \circled{2}\\ 
            
            6  & $\cos 2x + 3\sin x - 2 = 0$                                 &\\ 
            
            7  & $3\cos 2x - 5\sin x + 1 = 0$                                &\\ 
            
            8  & $\cos 2x - 5\sqrt{2}\cos x - 5 = 0$                         &\\ 
            \hline



            9   & $\cos (\frac{\pi}{2} + 2x) = \sqrt{2}\sin x$               & \circled{3}\\ 
            
            10   & $\cos 2x = \sin (x + \frac{\pi}{2})$                      &\\ 
            
            11  & $2\cos (\frac{\pi}{2} + x) = \sqrt{3}\tan x$               &\\ 
            \hline


            12   & $2\sin^ 4x + 3\cos 2x + 1 = 0$                            & \circled{4}\\ 
            
            13   & $4 \sin^4 2x + 3\cos 4x - 1 = 0$                          &\\ 
            
            14  & $4\cos^4 x - 4\cos^2 x + 1 = 0$                            &\\ 
            \hline


            15   & $(2\cos x + 1)(\sqrt{-\sin x} - 1) = 0$                   & \circled{5}\\ 
            
            16   & $(2\sin x - 1)(\sqrt{-\cos x} + 1) = 0$                   &\\ 
            
            17  & $\sqrt{\cos^2x - \sin^2 x}(\tan 2x - 1) = 0$               &\\ 
            \hline

            18  & $\cos^2 x - \frac{1}{2}\sin 2x + \cos x = \sin x$          & \circled{6}\\ 
            
            19  & $\frac{1}{2} \sin 2x + sin^2x - \sin x = \cos x$           &\\ 
            \hline


            20  & $\tan x + \cos (\frac{3\pi}{2} - 2x) = 0$                   & \circled{7}\\ 
            
            21  & $\cot x + \cos (\frac{\pi}{2} + 2x) = 0$                    &\\ 
            \hline

            22  & $\frac{2\sin^2 x - \sin x}{2\cos x - \sqrt{3}} = 0$         & \circled{8}\\ 
            
            23  & $\frac{2\sin^2 x - \sin x}{2\cos x + \sqrt{3}} = 0$         &\\ 
      \end{longtable}
      \end{center}
      \endgroup


      \begingroup
      \footnotesize
      \renewcommand\arraystretch{1.7}
      \begin{center}
      \begin{longtable}{|c|c|c|}
      \caption{Корпус, разбивающий уравнения по видам схожести подвыражений}\label{tbl:Corpus_subexprSim}\\
      \hline
      № & Выражение & Класс схож. подвыраж.\\
      \hline
      \endfirsthead
      \multicolumn{3}{l}%
      {Продолжение табл. \thetable\ } \\
      \hline
      № & Выражение & Класс схож. подвыраж.\\
      \hline
      \endhead
      \hline
      \endfoot
      \hline
      \endlastfoot
            1   & $\mB{\cos (\frac{\pi}{2}  + 2x)} = \sqrt{2}\sin x$          & \circled{1}\\ % 9
            
            2   & $\cot x + \mB{\cos (\frac{\pi}{2} + 2x)} = 0$              &\\             % 21
            \hline
            
            3  & $(2\cos x + 1)( \mB{\sqrt{-\sin x}}  - 1) = 0$                & \circled{2}\\ % 15
            
            4  & $(2\sin x - 1)(\mB{\sqrt{-\cos x}} + 1) = 0$                &\\             % 16
            \hline

            5  &  $\sqrt{2}\mB{\sin (\frac{3\pi}{2} - x)}\sin x = \cos x$    & \circled{3}\\ % 1

            6  &  $2\mB{\sin (\frac{7\pi}{2} - x)}\sin x = \cos x$           &\\             % 3
            \hline

            7  & $\mBH{\frac{2\sin^2 x - \sin x}{2\cos x - \sqrt{3}}} = 0$    & \circled{4}\\ % 22

            8  & $\mBH{\frac{2\sin^2 x - \sin x}{2\cos x + \sqrt{3}}} = 0$    &\\             % 23
            \hline
            % поменять их местами, чтобы не палевно было
            9  & $2\mB{\sin^4 x} + 3\cos 2x + 1 = 0$                         & \circled{5}\\ % 12

            10  &  $4\mB{\cos^4 x} - 4\cos^2 x + 1 = 0$                     &\\             % 14

            11  & $\mB{\cos^2 x} - \frac{1}{2}\sin 2x + \cos x = \sin x$     &\\             % 18

            12  & $\frac{2\mBL{\sin^2 x} - \sin x}{2\cos x - \sqrt{3}} = 0$   &\\             % 23       

            13  & $\frac{2\mBL{\sin^2 x} - \sin x}{2\cos x + \sqrt{3}} = 0$   &\\             % 22
      \end{longtable}
      \end{center}
      \endgroup

    Ввиду малого количества выражений в корпусах, алгоритм будет оцениваться лишь с точки зрения точности получаемых результатов. Для этого воспользуемся классической формулой оценки точности:

      \begin{equation}
        \label{eq:algo_accuracy}
        Accuracy = \frac{TP}{TP+FP}
      \end{equation}

    В формуле \ref{eq:algo_accuracy} под $TP$ (true positive) будем понимать количество истинно--положительных первых $k$ найденных уравнений из базы, которые принадлежат к классу запрашиваемого уравнения. $FP$ (false positive) --- количество ложно--положительных первых $t$ найденных уравнений из базы, которые принадлежат к классу запрашиваемого уравнения. На $k$ и $t$ вводится ограничение: $k + t = n-1$, где n --- количество уравнений, принадлежащих запрашиваемому уравнению.
    
    Данное ограничение было введено ввиду того, что классы схожести, определенные в корпусах, определяют самые схожие друг с другом уравнения, и не существует точной границы, относящие те или иные уравнения к тому же самому классу. 

    Ниже приведены примеры подсчета точности поиска алгоритма при поиске структурной схожести и схожести подвыражений на основе формулы \ref{eq:algo_accuracy} с учетом описанных выше особенностей.
    


  \section{Оценка эффективности алгоритма при поиске структурной схожести} \label{chapt4__efficiency__struct}
	

    Эффективность разработанного алгоритма при поиске структурной схожести оценивалась на одном из составленных корпусов (табл.~\ref{tbl:Corpus_structSim}), разделяющим уравнения по классам структурной схожести.

    На рис.~\ref{img:structSimSample} приведен пример подсчета структурной схожести между двумя уравнениями из базы.

      \begin{figure}[ht] 
        \center
        \includegraphics[width=\linewidth]{img/tikzpictures/chapt_4/structSimSample}
        \caption{Подсчет структурной схожести между уравнениями $\sqrt{2}\sin (\frac{3\pi}{2} - x)\sin x = \cos x$  и $2\cos (x - \frac{11\pi}{2})\cos x = \sin x$}
        \label{img:structSimSample}
      \end{figure}

    На рис.~\ref{img:structSimSample} мы видим что уравнения имеют 20 общих узлов друг с другом. Уравнение слева имеет 34 узла, а уравнение справа --- 33. С применение формулы \ref{eq:new_Sto} получаем $S_{TO} = \frac{20+20}{34+33} = \frac{40}{67} = 0.597$.    

    В табл.~\ref{tbl:structSimRes1} приведен результат поиска первых пяти структурно схожих уравнений из базы уравнению $\sqrt{2}\sin (\frac{3\pi}{2} - x)\sin x = \cos x$, принадлежащему к 1-му классу структурно схожих уравнений из корпуса (табл.~\ref{tbl:Corpus_structSim}).

        \begin{table} [htbp]
          \centering
          \caption{Результат поиска первых пяти структурно схожих уравнений из базы уравнению $\sqrt{2}\sin (\frac{3\pi}{2} - x)\sin x = \cos x$, принадлежащему к 1-му классу структурно схожих уравнений}
          \label{tbl:structSimRes1}
          \renewcommand\arraystretch{1.4}
          \begin{tabular}{| p{0.2cm} | p{4.4cm} | p{1.2cm} | p{1.2cm} | p{0.5cm}l |}
            \hline
            \centering № & \centering Выражение &\centering Кол--во общих узлов &\centering Схож. &\centering Вид&\\
            \hline
            1 &   $2\sin (\frac{7\pi}{2} - x)\sin x = \cos x$            & 60 / 67 & 0.896&\circled{1}&\\
            2 &   $2\cos (x - \frac{11\pi}{2})\cos x = \sin x$           & 40 / 67 & 0.597&\circled{1}&\\
            3 &   $2\cos (\frac{\pi}{2} + x) = \sqrt{3}\tan x$           & 24 / 63 & 0.381&\circled{3}&\\
            4 &   $3\cos 2x - 5\sin x + 1 = 0$                           & 12 / 60 & 0.200&\circled{2}&\\
            5 &   $\tan x + \cos (\frac{3\pi}{2} - 2x) = 0$              & 10 / 67 & 0.149&\circled{7}&\\
            \hline
          \end{tabular}
        \end{table}

    Поскольку запрашиваемое уравнение принадлежит 1-му классу, а всего уравнений в этом классе 4, то сумма количества истинно--положительных и количества ложно--положительных найденных алгоритмом уравнений не должно превышать $n-1=4-1=3$, поскольку мы ищем самые схожие уравнения.
    Точность алгоритма при поиске схожих уравнений 1-му уравнению 1-го класса из корпуса будет определяться по формуле \ref{eq:algo_accuracy} и равна:
    $Accuracy_{1}^{1} = \frac{TP}{TP+FP} = \frac{2}{2+1} = \frac{2}{3} = 0.66$.

    Причина того, почему в табл.~\ref{tbl:structSimRes1} не попало уравнение $-\sqrt{2}\sin (-\frac{5\pi}{2} + x)\sin x = \cos x$ несмотря на то, что оно также было отнесено 1-му классу схожести, будет объяснена ниже в подразделе \ref{chapt4__results} при анализе результатов.


    Вычислим точность поиска структурно схожих уравнений для каждого из уравнений корпуса (табл.~\ref{tbl:algo_struct_accuracy}).

    \begingroup
    \renewcommand\arraystretch{1.4}
    \begin{center}
    \begin{longtable}{|c|c|c|}
    \caption{Оценка точности структурного поиска}\label{tbl:algo_struct_accuracy}\\
    \hline
    № & Выражение & Точность\\
    \hline
    \endfirsthead
    \multicolumn{3}{l}%
    {Продолжение табл. \thetable\ } \\
    \hline
    № & Выражение & Точность\\
    \hline
    \endhead
    \hline
    \endfoot
    \hline
    \endlastfoot
          1   & $\sqrt{2}\sin (\frac{3\pi}{2} - x)\sin x = \cos x$         & 2/3\\ 
          
          2   & $2\cos (x - \frac{11\pi}{2})\cos x = \sin x$               & 2/3\\ 
          
          3  & $2\sin (\frac{7\pi}{2} - x)\sin x = \cos x$                 & 2/3\\ 
          
          4  & $-\sqrt{2}\sin (-\frac{5\pi}{2} + x)\sin x = \cos x$        & 0/3\\ 
          \hline
          

          5  & $\cos 2x - 3\cos x + 2 = 0$                                 & 2/3\\ 
          
          6  & $\cos 2x + 3\sin x - 2 = 0$                                 & 2/3\\ 
          
          7  & $3\cos 2x - 5\sin x + 1 = 0$                                & 1/3\\ 
          
          8  & $\cos 2x - 5\sqrt{2}\cos x - 5 = 0$                         & 2/3\\ 
          \hline


          9   & $\cos (\frac{\pi}{2} + 2x) = \sqrt{2}\sin x$               & 1/2\\ 
          
          10   & $\cos 2x = \sin (x + \frac{\pi}{2})$                      & 1/2\\ 
          
          11  & $2\cos (\frac{\pi}{2} + x) = \sqrt{3}\tan x$               & 0/2\\ 
          \hline


          12   & $2\sin^ 4x + 3\cos 2x + 1 = 0$                            & 1/2\\ 
          
          13   & $4 \sin^4 2x + 3\cos 4x - 1 = 0$                          & 1/2\\ 
          
          14  & $4\cos^4 x - 4\cos^2 x + 1 = 0$                            & 1/2\\ 
          \hline


          15   & $(2\cos x + 1)(\sqrt{-\sin x} - 1) = 0$                   & 2/2\\ 
          
          16   & $(2\sin x - 1)(\sqrt{-\cos x} + 1) = 0$                   & 2/2\\ 
          
          17  & $\sqrt{\cos^2x - \sin^2 x}(\tan 2x - 1) = 0$               & 2/2\\ 
          \hline

          18  & $\cos^2 x - \frac{1}{2}\sin 2x + \cos x = \sin x$          & 0/1\\ 
          
          19  & $\frac{1}{2} \sin 2x + sin^2x - \sin x = \cos x$           & 0/1\\ 
          \hline


          20  & $\tan x + \cos (\frac{3\pi}{2} - 2x) = 0$                   & 1/1\\ 
          
          21  & $\cot x + \cos (\frac{\pi}{2} + 2x) = 0$                    & 1/1\\ 
          \hline

          22  & $\frac{2\sin^2 x - \sin x}{2\cos x - \sqrt{3}} = 0$         & 1/1\\ 
          
          23  & $\frac{2\sin^2 x - \sin x}{2\cos x + \sqrt{3}} = 0$         & 1/1\\ 
    \end{longtable}
    \end{center}
    \endgroup

    По результатам табл.~\ref{tbl:algo_struct_accuracy}, точность нахождения структурно схожих уравнений из базы уравнениям из корпуса равна средней точности определения схожих уравнений из базы каждому уравнению из корпуса и равна 
    $Accuracy = \frac{\frac{2}{3}+\frac{2}{3}+\frac{2}{3}+\frac{0}{3}+ \dots+ \frac{1}{1}+\frac{1}{1}}{23} = \frac{13.85}{23} =0.6$.
    
    Причина низкой точности и способ ее увеличения при поиске структурной схожести данным алгоритмом будет объяснена при анализе результатов в параграфе \ref{chapt4__results__struct}.



  \section{Оценка эффективности алгоритма при поиске схожести подвыражений} \label{chapt4__efficiency__subexpr}


    Эффективность разработанного алгоритма при поиске схожести подвыражений оценивалась на одном из составленных корпусов (табл.~\ref{tbl:Corpus_subexprSim}), разделяющим уравнения по классам схожести подвыражений.

    На рис.~\ref{img:subexprSimSample} приведен пример подсчета схожести подвыражений между двумя уравнениями из базы.

      \begin{figure}[ht] 
        \center
        \includegraphics[width=.98\linewidth]{img/tikzpictures/chapt_4/subexprSimSample}
        \caption{Подсчет схожести подвыражений между уравнениями $\cos (\frac{\pi}{2} + 2x) = \sqrt{2}\sin x$  и $\ctg x + \cos (\frac{\pi}{2} + 2x) = 0$ }
        \label{img:subexprSimSample}
      \end{figure}


    На рис.~\ref{img:subexprSimSample} мы видим, что уравнения имеют 16 общих узлов друг с другом, образующих общее поддерево. Уравнение слева имеет 29 узлов, а уравнение справа --- 30. С применение формулы \ref{eq:new_Sto} получаем $S_{TO} = \frac{16+16}{29+30} = \frac{32}{59} = 0.542$.    

    В табл.~\ref{tbl:subexprSimRes9} приведен результат поиска первых пяти схожих на уровне подвыражений уравнений из базы уравнению $\cos (\frac{\pi}{2} + 2x) = \sqrt{2}\sin x$, принадлежащему к 1-му классу схожих на уровне подвыражений уравнений из корпуса (табл.~\ref{tbl:Corpus_subexprSim}).

        \begin{table} [htbp]
          \centering
          \caption{Результат поиска первых пяти схожих на уровне подвыражений уравнений из базы уравнению $2\sin^4 x + 3\cos 2x + 1 = 0$, принадлежащему к 5-му классу}
          \label{tbl:subexprSimRes9}
          \renewcommand\arraystretch{1.4}
          \begin{tabular}{| p{0.2cm} | p{4.8cm} | p{1.2cm} | p{0.9cm} | p{0.5cm}l |}
            \hline
            \centering № & \centering Выражение &\centering Кол--во общих узлов &\centering Схож. &\centering Класс схож. ур-я&\\
            \hline
            1 &   $4\cos^4 x - 4\cos^2 x + 1 = 0$                       & 10/59  & 0.169& \circled{5} &\\
            2 &   $4 \sin^4 2x + 3\cos 4x - 1 = 0$                      & 10/60  & 0.167& &\\
            3 &   $\cos^2 x - \frac{1}{2}\sin 2x + \cos x = \sin x$     & 10/63 & 0.159&  \circled{5} &\\
            4 &   $\frac{2\sin^2 x - \sin x}{2\cos x - \sqrt{3}} = 0$   & 10/64  & 0.156& \circled{5} &\\
            5 &   $\frac{2\sin^2 x - \sin x}{2\cos x + \sqrt{3}} = 0$   & 10/64  & 0.156& \circled{5} &\\
            \hline
          \end{tabular}
        \end{table}


    Поскольку запрашиваемое уравнение принадлежит 5-му классу, а всего уравнений в этом классе 5, то сумма количества истинно--положительных и количества ложно--положительных найденных алгоритмом уравнений не должно превышать $n-1=5-1=4$, поскольку мы ищем самые схожие уравнения.
    Точность алгоритма при поиске схожих уравнений 1-му уравнению 5-го класса из корпуса будет определяться по формуле \ref{eq:algo_accuracy} и равна:
    $Accuracy_{1}^{5} = \frac{TP}{TP+FP} = \frac{3}{3+1} = \frac{3}{4} = 0.75$.

    Причина того, почему уравнение $4 \sin^4 2x + 3\cos 4x - 1 = 0$ попало в (табл.~\ref{tbl:Corpus_subexprSim}) и повлияло на точность результатов, будет изложена ниже в параграфе \ref{chapt4__results__subexpr}.

    Подсчитаем точность поиска схожих уравнений на уровне подвыражений для каждого из уравнений корпуса (табл.~\ref{tbl:algo_subexpr_accuracy}).


      \begingroup
      \renewcommand\arraystretch{1.7}
      \begin{center}
      \begin{longtable}{|c|c|c|}
      \caption{Оценка точности схожести подвыражений}\label{tbl:algo_subexpr_accuracy}\\
      \hline
      № & Выражение & Точность\\
      \hline
      \endfirsthead
      \multicolumn{3}{l}%
      {Продолжение табл. \thetable\ } \\
      \hline
      № & Выражение & Точность\\
      \hline
      \endhead
      \hline
      \endfoot
      \hline
      \endlastfoot
            1   & $\cos (\frac{\pi}{2}  + 2x) = \sqrt{2}\sin x$          & 1/1\\ % 9
            
            2   & $\cot x + \cos (\frac{\pi}{2} + 2x) = 0$              & 1/1\\             % 21
            \hline
            
            3  & $(2\cos x + 1)( \sqrt{-\sin x}  - 1) = 0$              & 1/1 \\ % 15
            
            4  & $(2\sin x - 1)(\sqrt{-\cos x} + 1) = 0$                & 1/1\\             % 16
            \hline

            5  &  $\sqrt{2}\sin (\frac{3\pi}{2} - x)\sin x = \cos x$    & 1/1\\ % 1

            6  &  $\sin (\frac{7\pi}{2} - x)\sin x = \cos x$           & 1/1\\             % 3
            \hline

            7  & $\frac{2\sin^2 x - \sin x}{2\cos x - \sqrt{3}} = 0$    & 1/1\\ % 22

            8  & $\frac{2\sin^2 x - \sin x}{2\cos x + \sqrt{3}} = 0$    & 1/1\\             % 23
            \hline

            9  & $2\sin^4 x + 3\cos 2x + 1 = 0$                         & 4/5\\ % 12

            10  &  $4\cos^4 x - 4\cos^2 x + 1 = 0$                     & 4/5\\             % 14

            11  & $\cos^2 x - \frac{1}{2}\sin 2x + \cos x = \sin x$     & 4/5\\             % 18

            12  & $\frac{2\sin^2 x - \sin x}{2\cos x - \sqrt{3}} = 0$   & 5/5\\             % 23       

            13  & $\frac{2\sin^2 x - \sin x}{2\cos x + \sqrt{3}} = 0$   & 5/5\\             % 22
      \end{longtable}
      \end{center}
      \endgroup



    По результатам табл.~\ref{tbl:algo_subexpr_accuracy}, точность нахождения схожих на уровне подвыражений уравнений из базы уравнениям из корпуса равна средней точности определения схожих уравнений из базы каждому уравнению из корпуса и равна 
    $Accuracy = \frac{\frac{1}{1}+\frac{1}{1}+ \dots+ \frac{4}{5}+\frac{5}{5}+\frac{5}{5}}{13} = \frac{12.4}{13} = 0.95$.



  \section{Анализ результатов оценки эффективности алгоритма}  \label{chapt4__results}


    \subsection{Анализ результатов по оценке алгоритма при поиске структурной схожести} \label{chapt4__results__struct}


      В подразделе \ref{chapt4__efficiency__struct} была определена точность алгоритма при поиске структурной схожести и составила 0.6\%. Такой низкой точности послужила особенность в синтаксической конструкции выражений, представленных в формате MathML, по которым производился поиск структурной схожести. 


      Рассмотрим уравнения $\sqrt{2}\sin (\frac{3\pi}{2} - x)\sin x = \cos x$ и $-\sqrt{2}\sin (-\frac{5\pi}{2} + x)\sin x = \cos x$, принадлежащие к 1-му классу структурной схожести по разработанному корпусу. В результате поиска по разработанному алгоритму, схожести между ними замечено не было (рис.~\ref{img:structSimProblem}).

        \begin{figure}[ht] 
          \center
          \includegraphics[width=.7\linewidth]{img/tikzpictures/chapt_4/structSimProblem}
          \caption{Подсчет структурной схожести между уравнениями $\sqrt{2}\sin (\frac{3\pi}{2} - x)\sin x = \cos x$  и $-\sqrt{2}\sin (-\frac{5\pi}{2} + x)\sin x = \cos x$}
          \label{img:structSimProblem}
        \end{figure}

      Согласно паттерну структурной схожести выражений \ref{chapt1_simPatterns}, для того, чтобы выражения были структурно схожи, они должны иметь общие самые верхние узлы с одинаковым названием и одинаковыми правилами перехода к последующим узлам. У самых верхних узлов на рис.~\ref{img:structSimProblem} разное количество последующих узлов и, как следствие, разные правила перехода. Ввиду этого, несмотря на визуально схожую структуру и отнесению экспертами уравнений $\sqrt{2}\sin (\frac{3\pi}{2} - x)\sin x = \cos x$ и $-\sqrt{2}\sin (-\frac{5\pi}{2} + x)\sin x = \cos x$ к одному классу структурной схожести, в ходе выполнения поиска по разработанному алгоритму их структурная схожесть не определяется.

      На рис.~\ref{img:structSimSolveProblem} можно увидеть рассматриваемые уравнения в двух разных видах --- изначальном ($T_1$ и $T_2$) и имеющим другую структурную конструкцию ($T_1^{'}$ и $T_2^{'}$), но визуально и семантически представляемых одинаково. Тем не менее, их синтаксическая схожесть в изначальном представлении не будет определена, а в модифицированном будет равна 0.44. 

      \begin{sidewaysfigure}
        \center
        \includegraphics[width=.98\linewidth]{img/tikzpictures/chapt_4/structSimSolveProblem}
        \caption{Пример перехода от одних сравниваемых выражений к другим, имеющим другую синтаксическую структуру, но эквивалентным изначальным с точки зрения визуального представления и семантического смысла с целью поиска их синтаксической схожести}
        \label{img:structSimSolveProblem}
      \end{sidewaysfigure}

      С целью выявления синтаксической схожести выражений, имеющих разное количество узлов, начиная от коренных узлов (пример: выражения $T_1$ и $T_2$ на рис.~\ref{img:structSimSolveProblem}), данные выражения необходимо предварительно преобразовать (пример: выражения $T_1^{'}$ и $T_2^{'}$ на рис.~\ref{img:structSimSolveProblem}), выделяя необходимые конструкции в выражениях в отдельные подвыражения. Для этого предлагается один из следующих подходов:

      \begin{enumerate}
        \item выбрать другой конвертер для конвертации выражений из \LaTeX\ в формат MathML, который бы выделял  необходимые конструкции в отдельные подвыражения;
        \item при конвертации выражений из формата \LaTeX\ в формат MathML используемым конвертером \cite{mathmlconverter} нарочно добавлять фигурные скобки, ограничивающие конструкции, которые должны быть выделены в подвыражения: этот способ и был использован на рис.~\ref{img:structSimSolveProblem};\\
        
        \begin{itemize}
        \small{ \item \$ \bSL sqrt\{2\}  \bSL sin (\bSL frac\{3\bSL pi\}\{2\} - x) \bSL sin x = \bSL cos x \$ $\Rightarrow$ \\
          \$ \{\bSL sqrt\{2\}\}  \{\bSL sin (\bSL frac\{3\bSL pi\}\{2\} - x)\} \{\bSL sin x\} = \{\bSL cos x\} \$\\
          \item \$ -\bSL sqrt\{2\}  \bSL sin (\bSL frac\{5\bSL pi\}\{2\} + x) \bSL sin x = \bSL cos x \$ $\Rightarrow$ \\
          \$ \{-\bSL sqrt\{2\}\}  \{\bSL sin (\bSL frac\{5\bSL pi\}\{2\} + x)\} \{\bSL sin x\} = \{\bSL cos x\} \$\\}
        \end{itemize}

        \item производить анализ выражения, уже представленного в формате MathML, и по определенному алгоритму выделять необходимые конструкции в подвыражения: данный вариант является более перспективным в силу того, что уже существующие выражения, представленные в MathML, не нужно будет пересоздавать заново.
      \end{enumerate}

      После анализа особенности представления выражений в формате MathML были внесены поправки во все выражения из базы перед конвертацией их из \LaTeX\ в формат MathML с помощью 2-ого подхода. После этого структурный поиск при повторном тестировании на том же корпусе был абсолютно точен и составил 100\%.

    \subsection{Анализ результатов по оценке алгоритма при поиске схожести подвыражений} \label{chapt4__results__subexpr}


      На ухудшение точности поиска подвыражений в подразделе \ref{chapt4__efficiency__subexpr} повлияло уравнение, найденное вторым в табл.~\ref{tbl:Corpus_subexprSim}, которое не относилось к тому же классу схожести подвыражений запрашиваемого уравнения. Такой результат был получен ввиду того, что одно и то же уравнение может содержать несколько подвыражений, которые могут в свою очередь быть общими с подвыражениями разных уравнений. Рассмотрим запрашиваемое уравнение подробнее:
      \begin{itemize}
        \item $2\mB{\sin^4 x} + 3\boxed{\cos 2x} + 1 = 0$
      \end{itemize}

      Серым цветом обозначена область, которая повлияла на мнение экспертов, которые составляли корпус, чтобы отнести уравнение к определенному классу схожести подвыражений. В тоже время не было учтено то, что у того же самого уравнения есть другое подвыражение, выделенное в рамку, которое также может отнести уравнение и к другому классу схожести. При поиске и произошла подобная ситуация:
      \begin{itemize}
        \item $2\mB{\sin^4 x} + 3\boxed{\cos 2x} + 1 = 0$
        \item $4 \sin^4 2x + 3\boxed{\cos 4x} - 1 = 0$
      \end{itemize}

      Второе уравнение схоже с первым в подвыражении, выделенным в рамку, а не в подвыражении, выделенным серым цветом. В силу этой особенности точность поиска схожести подвыражений алгоритма на составленном корпусе составила 0.95\%. 




  \section{Общие выводы относительно полученных результатов и рекомендации по применению алгоритма} \label{chapt4__results__tips}


      Оценка точности поиска разработанным алгоритмом содержала ряд трудностей, изложенных в параграфах \ref{chapt4__results__struct} и \ref{chapt4__results__subexpr}. Тем не менее, после внесения ряда изменений с учетом этих особенностей, алгоритм показал высокую точность поиска структурной схожести и схожести подвыражений между заданным выражением и базой выражений (табл.~\ref{tbl:baseEGE}) на составленных корпусах (табл.~\ref{tbl:Corpus_structSim} и табл.~\ref{tbl:Corpus_subexprSim}).

      Полученные результаты относительно эффективности разработанного алгоритма можно считать успешными, и позволяют рекомендовать данный алгоритм к встраиванию в поисковые системы, нацеленные на работу с математической нотацией в формате MathML с разметкой Presentation Markup.


      Преимущества данного алгоритма перед существующими \cite{retr_tool__algo_subpath}, \cite{retr_tool__EMERS} по поиску синтаксической схожести между математическими выражениями, представленными в формате MathML:
        \begin{itemize}
          \item разделяет синтаксическую схожесть на подвиды (структурная схожесть и схожесть подвыражений) и позволяет ее находить;
          \item позволяет получить дополнительную информацию о схожих выражениях, такую как: кол-во общих узлов, коренной узел общей области, является ли эта общая область поддеревом;
          \item производит предварительную индексацию корпуса деревьев, по которым будет производиться поиск с целью ускорения поиска, которая также присутствует только в одной \cite{retr_tool__algo_subpath} из двух исследованных реализаций.
        \end{itemize}


      Можно привести следующие направления для дальнейшего развития алгоритма:
        \begin{itemize}
          \item введение весов на узлы деревьев выражений для выделения значимости тех или иных областей при поиске схожести подвыражений;
          \item расширение области его применения до поиска схожести между математическими выражениями, представленными в формате MathML с разметкой Content Markup;
          \item визуальное представление схожести выражений при отображении выражений.
        \end{itemize}
        




