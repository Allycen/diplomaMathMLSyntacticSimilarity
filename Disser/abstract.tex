
\keywords{%
  MATHML,
  СХОЖЕСТЬ МАТЕМАТИЧЕСКИХ ВЫРАЖЕНИЙ,
  СИНТАКСИЧЕСКАЯ СХОЖЕСТЬ,
  СТРУКТУРНАЯ СХОЖЕСТЬ,
  СХОЖЕСТЬ ПОДВЫРАЖЕНИЙ
}

\abstractcontent{
	Целью данной работы является исследование существующих решений и разработка новых алгоритмов по поиску синтаксической схожести между математическими выражениями, представленными в формате MathML.
	В настоящее время существует лишь один алгоритм, получивший широкое применение в области оценки систем по распознаванию рукописной математической нотации, который предоставляет возможность определить дистанцию редактирования, чтобы превратить одно выражение в другое, и не предоставляет никакой дополнительной информации о схожести двух выражений.

	В первой главе содержатся общие сведенья о способах определения схожести между упорядоченными ориентированными графами.

	Во второй главе дается определение паттернам синтаксической схожести выражений в формате MathML и приводится модификация одного из исследованных алгоритмов для определения схожести между выражениями.

	В третьей главе описан процесс разработки приложения, реализующего поиск схожести математических выражений на основе разработанного алгоритма.

	В четвертой главе приведены общие сведения о полученных результатах, приведена оценка разработанного алгоритма и рекомендации к его применению.
}

\keywordsen{
  MATHML,
  SYNTACTIC SIMILARITY,
  STRUCTURAL SIMILARITY,
  SUBEXPRESSION SIMILARITY,
}

\abstractcontenten{
  The aim of this work is to study existing solutions and to develop new algorithms for searching the syntactic commonality between the mathematical expressions presented in the MathML format. Nowadays there is only one widely used algorithm in the evaluation systems of handwritten mathematical notation recognition, which provides the possibility to define the edit distance to transform one expression into another and doesn`t contribute any additional information about the commonality of two expressions. 

  The first chapter contains general information about the ways of the commonality determination between the ordered directed graphs. 

  The second chapter defines the syntactical commonality patterns of the expressions in MathML format and provides the modification of the investigated algorithm for determining commonality between the expressions. 

  The third chapter describes the process of an application development that implements the search of the mathematical expressions commonality on the basis of the developed algorithm. 
  
  The fourth chapter presents the results overview, the evaluation of developed algorithm and recommendations for use.
}
