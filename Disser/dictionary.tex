\chapter*{СЛОВАРЬ ТЕРМИНОВ}
\addcontentsline{toc}{chapter}{СЛОВАРЬ ТЕРМИНОВ} 



\textbf{XML (от англ.~ eXtensible Markup Language)} --- расширяемый язык разметки документов, используемых в сети интернет.


\textbf{MathML (от англ.~Mathematical Markup Language)} --- язык разметки на основе XML для представления математических символов и формул в документах сети интернет.


\textbf{EMERS (от англ.~a tree matching-based performance evaluation metric for mathematical expression recognition)} --- метрика определения схожести распознанного математического выражения с действительным. 


\textbf{Нотация} --- система условных обозначений, принятая в какой-либо области знаний или деятельности. Включает множество символов, используемых для представления понятий и их взаимоотношений, составляющее алфавит нотации, а также правила их применения.


\textbf{Нормализация математического выражения} --- процесс удаления или замены некоторой неважной информации в древовидном представлении математического выражения, такой как, к примеру, название переменных или численных значений.


\textbf{Терминальный узел} —-- узел, не имеющий дочерних элементов.


