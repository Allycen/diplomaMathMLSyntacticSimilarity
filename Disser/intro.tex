%%%%%%%%%%%%%%%%%%%%%%%%%%%%%%%%%%%%%%%%%%%%%%%%%%%%%%%%%%%%%%%%%%%%%%%%%%%%%%%%
\intro
%%%%%%%%%%%%%%%%%%%%%%%%%%%%%%%%%%%%%%%%%%%%%%%%%%%%%%%%%%%%%%%%%%%%%%%%%%%%%%%%

В наши дни поисковые системы в интернете являются неотъемлемыми и необходимыми инструментами. К сожалению, поиск математической нотации до сих пор сопряжен с рядом трудностей. Они состоят и в нетривиальной математической разметке математических выражений и формул, затрудняющей формулировку запроса пользователем, и в отсутствии унифицированной цифровой математической библиотеки. До сих пор нет общепринятого соглашения о том, какой формат хранения математической нотации использовать. Из--за этого многие поисковые машины сталкиваются с проблемой конвертации различных форматов в один.

Один из наиболее распространенных языков для представления математических выражений в интернете является MathML, рекомендованный математической группой W3C. Многие поисковые системы используют этот язык для хранения математической нотации. В основе практически всех таких систем лежит поиск по точному совпадению между выражениями или их частями, и не предусматривается или слабо развит поиск схожих математических выражений. Из всех рассмотренных алгоритмов поиска схожих математических выражений \cite{retr_tool__algo_subpath}, \cite{retr_tool__EMERS}, каждый имеет ряд существенных недостатков в виде плохой точности поиска по большой базе выражений или в виде нацеленности на решение узкого круга задач.

Синтаксическая схожесть математических выражений до сих пор определена нечетко --- нет общепринятых паттернов, которые определяли бы схожесть двух выражений, полагаясь на те или иные критерии. В то же время разработка поисковой системы, которая реализовала бы возможность подобного рода поиска, была бы очень актуальна при сортировке однотипных заданий в базе заданий единых государственных экзаменов ОГЭ и ЕГЭ или поиске похожих заданий из сторонних учебных ресурсов с целью того, чтобы учащийся или абитуриент мог систематичнее и тщательнее подготовиться к заданиям определенного вида. Кроме этого, система могла бы найти полезное применение при формировании электронных методических пособий или поиску выражений по шаблону, когда пользователь не знает точного выражения, но помнит его форму, синтаксическую структуру.


Целью данной работы является разработка алгоритма определения синтаксической схожести выражений в формате MathML и его реализация в виде программного комплекса для последующей оценки его эффективности.