%%%%%%%%%%%%%%%%%%%%%%%%%%%%%%%%%%%%%%%%%%%%%%%%%%%%%%%%%%%%%%%%%%%%%%%%%%%%%%%%
\conclusion
%%%%%%%%%%%%%%%%%%%%%%%%%%%%%%%%%%%%%%%%%%%%%%%%%%%%%%%%%%%%%%%%%%%%%%%%%%%%%%%%

В данной работе определены паттены синтаксической схожести между выражениями, представленными в формате MathML. Был модифицирован существующий алгоритм нахождения схожести между упорядоченными ориентированными деревьями для адаптации его к поиску синтаксически схожих математических выражений. Было разработано программное обеспечение, реализующее поиск схожих математических выражений на основе модифицированного алгоритма. 

Для проверки эффективности разработанного алгоритма, тестирование полученных результатов проводилось на корпусах, составленных, из уравнений, взятых с сайта для подготовки абитуриентов к сдаче единого государственного экзамена. 
В ходе тестирования обнаружилилась особенность в инвариантности древовидного представления математических выражений после их конвертации из формата \LaTeX\ в формат MathML. В силу этого, при определенной структуре тестируемых выражений, алгоритм демострирует средние результаты по поиску, в остальных случаях алгоритм с высокой степенью точности определяет схожие выражения. Анализ результатов тестирования позволил выявить направления для дальнейшего совершенствования алгоритма. 

К отличительным особенностям данного алгоритма можно отнести возможность задавать тип схожести (структурная схожесть или схожесть подвыражений) при поиске синтаксической схожести между выражениями.

Разработанный алгоритм может применяться в системах автоматизации сортировки схожих математических выражений для пополнения их к уже существующей базе примеров, системах по составлению тестовых заданий или задачников по математике, поиске схожих математических выражений по шаблону.

