
\documentclass[%
  a5paper,
  subf,
  href,
  master,
  dotsinheaders
]{csse-fcs}

\usepackage[T2A]{fontenc}
\usepackage[utf8]{inputenc}
\usepackage[english,russian]{babel}

% Компактные списки
\usepackage{mdwlist}

% Таблицы
\usepackage{array}

% Листинги
\usepackage{listings}

% Листинги
\usepackage{longtable}

% TODOs
\usepackage[%
  colorinlistoftodos,
  shadow
]{todonotes}

% Путь к каталогу со всеми рисунками
\graphicspath{{fig/}}

% Автоконвертер для EPS
\usepackage{epstopdf}

% Генератор текста
\usepackage{blindtext}

%%%%%%%%%%%%%%%%%%%%%%%%%%%%%%%%%%%%%%%%%%%%%%%%%%%%%%%%%%%%%%%%%%%%%%%%%%%%%%%%

% многострочные подчеркивания текста
\usepackage{soul}

% книжные таблицы
\usepackage{booktabs}

\newcommand*\circled[1]{\tikz[baseline=(char.base)]{
            \node[shape=circle,draw,inner sep=2pt] (char) {#1};}}

% для закрашивания квадратика в mathmode
\usepackage{xcolor}
\usepackage{etoolbox}

      \newcommand\mB[2][]{\tikz[baseline=(char.base)]\node[minimum width=0.5em,text height=1.3ex, text depth=0.3ex,fill=black!15,text=black,#1](char){$#2$};}%
      \newcommand\mBH[2][]{\tikz[baseline=(char.base)]\node[minimum width=2em,text height=1.7ex, text depth=0.5ex,fill=black!15,text=black,#1](char){$#2$};}%
      \newcommand\mBL[2][]{\tikz[baseline=(char.base)]\node[minimum width=0.4em,text height=1.0ex, text depth=0.1ex,fill=black!15,text=black,#1](char){$#2$};}%

% для поворота картинки и надписи на 90 градусов
\usepackage{rotating}

% backslash в math моде
\newcommand{\bSL}{\texttt{\textbackslash}}


% чтобы не было лишних остступов в листинге
\usepackage{lstautogobble}
