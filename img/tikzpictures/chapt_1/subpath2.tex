\documentclass{standalone}
\usepackage{tikz}
\usepackage{tikz-qtree}
\usepackage[makeroom]{cancel}
\usepackage[T2A,T1]{fontenc}
\usepackage[utf8]{inputenc}
\usepackage[english,russian]{babel}
\usetikzlibrary{fit}


\begin{document} 
	\begin{tikzpicture}
	\tikzset{level 1+/.style={level distance=2.5\baselineskip}}

	    \node (x) at (-0.8,0.5) {$T_0$};
	    \Tree [.$a_1^0$
	            [.$b_1^0$
	                [.$d_1^0$ ]
	                [.$e_1^0$ ] 
	            ] 
	            [.$c_1^0$ ]
	          ]
	    

	    \begin{scope}[xshift=2.2cm]
	    \node (y) at (-0.8,0.5) {$T_2$};
	    \tikzset{edge from parent/.style={very thick,draw}}
	    \Tree [.$a_1^2$
	            [.$g_1^2$
	                [.$i_1^2$ ]
	            ]  
	            [.$b_1^2$
	            	[.$d_1^2$ ]
	            	[.$e_1^2$ 
	            		[.$g_2^2$ 
	            			[.$j_1^2$ ]
	            		]
	            	]
	            ]
	          ]
	    \end{scope}


	    \begin{scope}[xshift=8.2cm, yshift=-1.7cm]

		\def\arraystretch{1.6}

	    \node (common) at (0,0) 
	    {
			\begin{tabular}{l}
				$(a),(b),(d),(e)$ \\
				$(a,b), (b,d), (b,e)$ \\
				$(a,b,d), (a,b,e)$ \\
				\\
				\\
			\end{tabular}
	    };

		\node (t1) at (-3.4,0) 
	    {
			\begin{tabular}{l}
				$(c)$ \\
				$(a,c)$ \\
				\\
				\\
				\\
			\end{tabular}
	    };


		\node (t2) at (4,0) 
	    {
			\begin{tabular}{l}
				$(g),(i), (j)$ \\
				$(a,g), (g,i), (e,g),(g,j)$ \\
				$(a,g,i), (b,e,g), (e,g,j)$ \\
				$(a,b,e,g), (b,e,g,j)$ \\
				$(a,b,e,g,j)$ \\
			\end{tabular}
	    };
	    % подгоняем размеры и расположение квадратиков
	    \newcommand{\xF}{-1.6}
	    \newcommand{\yF}{-1.9}
	    \newcommand{\width}{7.6}
	    \newcommand{\height}{3.7}
		\newcommand{\offsetX}{-2.5}
		\newcommand{\offsetY}{0.1}

		\newcommand{\xS}{\xF + \offsetX}
	    \newcommand{\yS}{\yF + \offsetY}

	    \draw [very thick](\xF,\yF) rectangle (\xF + \width, \yF + \height);
	    \draw [](\xS,\yS) rectangle (\xS + \width -2.0, \yS + \height);



	    \node (label1) at (-1.1,2.2) {Подпути дерева $T_0$};
	    \node (label2) at (2,-2.2) {Подпути дерева $T_2$};
	    \end{scope}

	\end{tikzpicture}
\end{document} 